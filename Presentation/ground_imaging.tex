\documentclass[11pt]{beamer}
\usetheme{Szeged}
\usepackage[utf8]{inputenc}
\usepackage{amsmath}
\usepackage{amsfonts}
\usepackage{amssymb}
\usecolortheme{dolphin}
\author{Kyle Colton\\Thomas Kwak}
\title{Classification of Land Types Through Clustering}
%\setbeamercovered{transparent} 
%\setbeamertemplate{navigation symbols}{} 
%\logo{}
\usepackage{animate}
\institute{Math 191} 
\date{\today}
\subject{Indian Pines Hyperspectral Imaging}
\begin{document}

\begin{frame}
\titlepage
\end{frame}

%\begin{frame}
%\tableofcontents
%\end{frame}
\section{Introduction}
\subsection{Problem}
\begin{frame}{Data Set}
The \texttt{indian\_pines} dataset consists of hyperspectral earth images of farmland
\begin{columns}[T]
\begin{column}{.48\textwidth}
% LEFT COLUMN
\begin{itemize}
\item Dimensions $145 x 145 x 200$
\item Each image is $145x145$
\item Images were captured at 200 different wavelengths
\end{itemize}
\end{column}
\hfill
\begin{column}{.48\textwidth}
% RIGHT COLUMN
\includegraphics[scale=.3]{gt.png}
\end{column}
\end{columns}
\end{frame}

\subsection{Exploration}
\begin{frame}{Exploring the Data}
\begin{center}
% To run this, you must first convert the GIF to a series of PNGs
% bash: convert indian_pines.gif indian_pines_%d.png
\animategraphics[autoplay,loop,height=6cm]{10}{indian_pines_}{0}{199}
\end{center}
\end{frame}

\section{Analysis}

%%%%%%%%%%%%%%%%%%%
%%%   K-Means   %%%
%%%%%%%%%%%%%%%%%%%

\section{K-Means}
\subsection{KM2}
\begin{frame}{K-Means}
\begin{columns}[T]
\begin{column}{.48\textwidth}
% LEFT COLUMN
\includegraphics[scale=.3]{km2.png}
\end{column}
\hfill
\begin{column}{.48\textwidth}
% RIGHT COLUMN
\includegraphics[scale=.3]{gt.png}
\end{column}
\end{columns}
\end{frame}

\subsection{KM5}
\begin{frame}{K-Means}
\begin{columns}[T]
\begin{column}{.48\textwidth}
% LEFT COLUMN
\includegraphics[scale=.3]{km5.png}
\end{column}
\hfill
\begin{column}{.48\textwidth}
% RIGHT COLUMN
\includegraphics[scale=.3]{gt.png}
\end{column}
\end{columns}
\end{frame}

\subsection{KM10}
\begin{frame}{K-Means}
\begin{columns}[T]
\begin{column}{.48\textwidth}
% LEFT COLUMN
\includegraphics[scale=.3]{km10.png}
\end{column}
\hfill
\begin{column}{.48\textwidth}
% RIGHT COLUMN
\includegraphics[scale=.3]{gt.png}
\end{column}
\end{columns}
\end{frame}

\subsection{KM16}
\begin{frame}{K-Means}
\begin{columns}[T]
\begin{column}{.48\textwidth}
% LEFT COLUMN
\includegraphics[scale=.3]{km16.png}
\end{column}
\hfill
\begin{column}{.48\textwidth}
% RIGHT COLUMN
\includegraphics[scale=.3]{gt.png}
\end{column}
\end{columns}
\end{frame}

%%%%%%%%%%%%%%%
%%%   SVM   %%%
%%%%%%%%%%%%%%%

\section{SVM}
\begin{frame}{Support Vector Machine}

\end{frame}

\section{Further Analysis}
\subsection{Noise Detection}
\begin{frame}{Future Ideas}
\begin{columns}[T]
\begin{column}{.48\textwidth}
% LEFT COLUMN
\includegraphics[scale=.4]{wavelength1.png}
\end{column}
\hfill
\begin{column}{.48\textwidth}
% RIGHT COLUMN
\begin{itemize}
\item Some of the data appears noisy
\item This could negatively impact classification
\item Detecting noise and biasing the model towards ``clean'' data could help the prediction
\end{itemize}
\end{column}
\end{columns}
\end{frame}

\begin{frame}{Historgrams}
\begin{center}
% To run this, you must first convert the GIF to a series of PNGs
% bash: convert histograms.gif histograms_%d.png
\animategraphics[autoplay,loop,height=6cm]{10}{histograms_}{0}{199}
\end{center}
\end{frame}

\section{Conclusion}
\subsection{Results}
\begin{frame}{Results}

\end{frame}

\subsection{Conclusion}
\begin{frame}{Conclusion}

\end{frame}
\end{document}
